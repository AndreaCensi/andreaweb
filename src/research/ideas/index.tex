\documentclass{article}

% Packages required to support encoding
\usepackage{ucs}
\usepackage[utf8x]{inputenc}

% Packages required by code


% Packages always used
\usepackage{hyperref}
\usepackage{xspace}
\usepackage[usenames,dvipsnames]{color}
\hypersetup{colorlinks=true,urlcolor=blue}



\begin{document} 
\hypertarget{timeline}{}\section*{{Timeline}}\label{timeline}

\begin{enumerate}%
\item Up to mid-April:

\begin{itemize}%
\item finish first draft of journal version for \href{http://purl.org/censi/2006/accuracy}{http\char58\char47\char47purl\char46org\char47censi\char47\char50\char48\char48\char54\char47accuracy}.
\item coordinate with people in Rome for journal version of \href{http://purl.org/censi/2007/calib}{work in calibration}.
\item talk with potential advisors (\href{http://www.cds.caltech.edu/~andrea/blog/2008-02-21}{list of people}).

\end{itemize}

\item May:

\begin{itemize}%
\item prepare for ICRA conference

\end{itemize}


\end{enumerate}
\hypertarget{themes}{}\section*{{Themes}}\label{themes}

\hypertarget{estimation}{}\subsection*{{Estimation}}\label{estimation}

\begin{itemize}%
\item Caratterizzare la covarianza in caso vincolato.

\begin{itemize}%
\item applicazione: calibration (cos{\tt \char94}2+sin{\tt \char94}2=1)

\end{itemize}

\item Studiare il bias.

\begin{itemize}%
\item applicazione: stima bias ICP

\end{itemize}

\item Write a paper about Cramer-Rao bounds for scan matching; that would be the extension of \href{http://purl.org/censi/2006/accuracy}{\emph{Cramer-Rao bounds for localization}}

\begin{itemize}%
\item main results already worked out and experimentally verified (= don'{}t bother with ICP).
\item countour examples to be worked out.

\end{itemize}


\end{itemize}
\hypertarget{planning_with_uncertainty}{}\subsection*{{Planning with uncertainty}}\label{planning_with_uncertainty}

\begin{itemize}%
\item Aggiungere trick celle.
\item Fare versione journal.

\end{itemize}
Caso continuo:

\begin{itemize}%
\item formalizzare caso continuo


\item ristudiare controllo ottimo

\begin{itemize}%
\item fare prima caso senza incertezza
\item ipotesi: rallento solo se l'{}informazione sta diminuendo.

\end{itemize}


\end{itemize}
\hypertarget{calibration}{}\subsection*{{Calibration}}\label{calibration}

\begin{itemize}%
\item Fare

\end{itemize}
\hypertarget{communication}{}\subsection*{{Communication}}\label{communication}

\begin{itemize}%
\item Rewrite in English

\end{itemize}
\hypertarget{cose_da_studiare}{}\section*{{Cose da studiare}}\label{cose_da_studiare}

\begin{itemize}%
\item rivedersi bene controllo ottimo
\item studiare Markov Random Fields
\item finire libro su Gaussian Process

\end{itemize}

\end{document}
